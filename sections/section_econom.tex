\section{Економічне обгрунтування науково-дослідницької роботи}
    \subsection{Вступ}

В сучасному світі майже неможливо ефективно розповсюджувати товар, якщо ринок збуту досяг великих розмірів. Це сприявло появі низки програмних продуктів, що дозволяють аналізувати діяльність маркетингових каналів, моделювати їх.

Система керування маркетинговою грою <<Маркетинговий канал>> допоможе організаторам маркетингових каналів здійснювати маркетингову розвідку шляхом моделювання маркетингового каналу. 

Ефективна маркетингова розвідка дозволяє маркетинговому каналу функціонувати більш оптимально, швидко реагувати на зміни зовнішньої та внутрішньої середи, оптимально керувати структурої каналу: виробництвом та посередниками. Ефективність функціонування каналу та оптимальність його структури безпосередньо позначаються на цінах для кінцевого споживача, що робить маркетинговий канал затребуваним.

\subsection{Обгрунтування мети і завдання дослідження}
Дана науково-дослідницька робота «Розробка моделей, алгоритмічного та програмного забезпечення для керування бізнес-грою «Маркетинговий канал».
Виділимо набір завдань, сукупність вирішень яких, на практичному рівні, приведе до досягнення бажаного результату:

\begin{enumerate}
\item вивчення технічної літератури по заданій предметній області;
\item забезпечення необхідного і достатнього рівня апаратного і програмного забезпечення;
\item підбір кваліфікованих фахівців для розробки даної інформаційної технології, в рамках бюджетних обмежень, фінансових ресурсів, що накладаються обмеженістю;
\item визначення оптимальних термінів виконання робіт, при яких проект збереже бажану економічну ефективність.
\end{enumerate}

Рішення, представлене в даній науково-дослідницькій роботі, не є першим у своєму роді, існують інші системи соціального обліку, але іх кількість можно порахувати на пальцях однієї руки і вони закриті, т.є використовуються для збору статистики. За основу береться стандартний набір функцій телефону – камера, інтеграція с соціальними мережами та можливість використовувати Інтернет.
    \subsection{Оцінка рівня науково-технічного ефекту роботи}
Визначення рівня науково-технічного ефекту НДР проводиться по бальних оцінках. За допомогою експертів встановлюється перелік основних чинників, що визначають науково-технічний рівень НДР. Кожен чинник характеризується декількома станами.

Експертами встановлюються оцінка в десятибальній системі кожного стану. Крім того, ними ж встановлюється і коефіцієнти ваги кожного чинника. Загальна оцінка рівня науково-технічного ефекту ($U_{\text{НДР}}$) визначається за наступною формулою:
  
\begin{equation}
U_{\text{НДР}} =  \frac{\sum \limits_{i} Q_i   \cdot   K}{\sum \limits_{i} Q_{mi}   \cdot   K}, \nonumber
\end{equation}
\begin{formulaDescription}
\item [\mbox{$ Q_i $}] оцінка науково-технічної значущності чинника в балах;
\item [\mbox{$ Q_{mi} $}] максимальна оцінка чинника;
\item [\mbox{$ K_i $}] коефіцієнт ваги даного чинника для науково-технічної ефективності НДР.
\end{formulaDescription}  


Для виконуваної роботи мають місце наступні чинники рівня науково-технічного ефекту НДР (кожен чинник має вагу і бальну оцінку):

Ступінь новизни. Дана робота є винаходом, що характеризується частковою новизною, має прототип, співпадаючий з новим рішенням:

\begin{equation}
Q_i = 6; Q_{mi} = 10; K_i = 10\%. \nonumber
\end{equation}

Рівень отриманого результату. Отримання нових матеріалів, речовин і т.п. серед аналогічних відомих видів:

\begin{equation}
Q_i = 4; Q_{mi} = 10; K_i = 22\%. \nonumber
\end{equation}

Ступінь теоретичної обґрунтованості результатів НДР. Завдання вирішене на основі застосування окремих пізнаних закономірностей:

\begin{equation}
Q_i = 4; Q_{mi} = 10; K_i = 5\%. \nonumber
\end{equation}

Ступінь експериментальної перевірки отриманих результатів. Експериментальна перевірка отриманих результатів не проводилася.

\begin{equation}
Q_i = 1; Q_{mi} = 6; K_i = 20\%. \nonumber
\end{equation}

Трудомісткість виконання НДР. Отримання результатів супроводжувалося проведенням нескладних дослідів, розрахунків, обґрунтувань.

\begin{equation}
Q_i = 2; Q_{mi} = 8; K_i = 10\%. \nonumber
\end{equation}

Перспективність роботи. Важливі результати сприяють задоволенню знов виникаючих потреб:

\begin{equation}
Q_i = 5; Q_{mi} = 10; K_i = 10\%. \nonumber
\end{equation}

Рівень досягнення світових стандартів. Дана робота на рівні світових стандартів:

\begin{equation}
Q_i = 7; Q_{mi} = 10; K_i = 8\%. \nonumber
\end{equation}

Рівень реалізації по об'ємах і термінах. Реалізація на рівні підприємства протягом до 3 років:

\begin{equation}
Q_i = 8; Q_{mi} = 10; K_i = 15\%. \nonumber
\end{equation}

Рівень науково-технічного ефекту дослідження при вище відмічених результатах складе:

\begin{equation}
U_{\text{ндр}} = \frac{6  \cdot  10+4  \cdot  22+4  \cdot  5+1  \cdot  20+7  \cdot  8+8  \cdot  15}{10  \cdot  10+10  \cdot  22+8  \cdot  10+10  \cdot  10+10  \cdot  8+10  \cdot  15} = 0,482(2). \nonumber
\end{equation}
    \subsection{Розрахунок кошторису витрат на проведення науково-дослідної роботи в лабораторних умовах}

Економічні показники науково-дослідної роботи розраховуються як показники роботи, що виконується в лабораторних умовах. Витрати на проведення науково-дослідних робіт відносять до виробничих витрат.

До складу науково-дослідних робіт включають:
\begin{itemize}
\item патентний пошук;
\item вивчення літератури;
\item розробка програми дослідження;
\item збір первинної інформації;
\item тестування машинних програм;
\item розрахункові роботи;
\item розробка креслень і схем;
\item виготовлення дослідного зразка (програми прототипу);
\item оформлення записки пояснення.
\end{itemize}

Плановий кошторис витрат складається по укрупнених статтях витрат:

Заробітна плата персоналу. Заробітна плата персоналу, що бере участь у виконанні науково-дослідної роботи, визначається на основі штатно-окладної форми оплати праці. Початкові і розрахункові показники зводяться в таблицю \ref{tbl:table61}.


\begin{stdtablelong}{6}{|C{3cm}|C{2cm}|C{2cm}|C{2cm}|C{2cm}|C{2cm}|}
{\label{tbl:table61}Витрати на заробітну плату}
{  
Склад виконавців &
%\multicolumn{3}{C{12cm}|}{Значення параметрів технологічного встаткування} &
Кількість працівників &
Місячний оклад, грн. &
Час роботи, міс. &
Коеф. участі в роб. &
Сума зарплати
}
Керівник роботи & 1 & 16000 & 6 & 1 & 96000 \\ \hline
Інженер-програміст & 1 & 6400 & 6 & 1 & 38400 \\ \hline
Лаборант & 1 & 2400 & 6 & 0,5 & 7200 \\ \hline
Разом & 3 & 24800 & 15 & 2.75 & 141600 \\ \hline
\end{stdtablelong}
Преміальний фонд приймається у розмірі 7\% від фонду заробітної плати і складає 9912 грн.

Відрахування до бюджету. 
На заробітну плату з урахуванням преміального фонду нараховуються відрахування до бюджету держави. До складу цих відрахувань включаються:
\begin{enumerate}
\item відрахування до пенсійного фонду – 33,2\%;
\item відрахування до фонду соціального страхування – 1,5\%;
\item відрахування до фонду зайнятості – 1,4\%;
\item відрахування до фонду страхування нещасних випадків – 0,8\%.
\end{enumerate}

Загальна сума відрахувань складе 36,9\% від фонду оплати праці, тобто \mbox{$(141600+9912)  \cdot  0,369 = 55907,928$ грн}.

Витрати на відрядження.

Витрати на науково-виробничі відрядження плануються у розмірі 15\% від фонду заробітної плати, тобто \mbox{$(141600+9912)  \cdot  0,15 = 22726.8$} грн.

Контрагентські витрати.

У кошторис витрат включаються витрати на послуги, здійснюваних по договорах. До таких послуг відносяться:
\begin{enumerate}
\item надання машинного часу обчислювального центру або персонального комп'ютера;
\item створення машинної бази даних;
\item виготовлення дослідних зразків;
\item розмноження оригіналів;
\item виготовлення графічних матеріалів і тому подібне.
\end{enumerate}

При оренді машинного часу на персональному комп'ютері передбачаються витрати у розмірі 10-15 грн. за кожну годину роботи. Потреба в машинному часі впродовж 6 місяців по 8 годин в день складає 6 $ \cdot $ 20 $ \cdot $ 8=960 годин, тобто витрати на оренду машинного часу складуть 960 $ \cdot $ 13=12480 грн.

Витрати на матеріали. Витрати на матеріали, канцелярсько-письмове приладдя розраховується по кількості і їх прейскурантним цінам. Перелік використовуваних матеріалів, потреба в них і їх ціни зводяться в таблицю \ref{tbl:table62}.

\begin{stdtablelong}{5}{|C{4cm}|C{2cm}|C{2cm}|C{2cm}|C{2cm}|}
{\label{tbl:table62}Витрати на матеріали}
{  
Найменування матеріалів &
Одиниці виміру &
Кількість &
Ціна, грн. &
Сума, грн. 
}
Папір            & Упаковка & 3  & 50,00 & 150,00 \\ \hline
Олівець простий  & Шт.      & 1  & 1,50  & 1,50  \\ \hline
Ручка синя       & Шт.      & 3  & 2,50  & 5,00  \\ \hline
Лінійка          & Шт.      & 1  & 3,50  & 3,50  \\ \hline
Гумка            & Шт.      & 1  & 2,50  & 2,50  \\ \hline
Ручка чорна      & Шт.      & 1  & 2,00  & 2,00  \\ \hline
Папір А1         & Шт.      & 8  & 10,00 & 80,00  \\ \hline
Разом            &          & 17 & 72,00 & 344,50 \\ \hline
\end{stdtablelong}

Витрати на електроенергію. 

Витрати на електроенергію розраховуються по потужності електроустановок. У перелік електроустановок слід включити:
\begin{enumerate}
\item прилади освітлення лабораторії;
\item нагрівальні установки;
\item випробувальні стенди;
\item вимірювальні прилади.
\end{enumerate}

Витрати на електроенергію по обчислювальній техніці, що орендується, в кошторис не включаються. Вони входять у вартість 1 години машинного часу. Витрати на електроенергію Зэ розраховуються по формулі:

\begin{equation}
Z =  \sum \limits_{h} W_h   \cdot   T_h   \cdot   K_h   \cdot   C_e, \nonumber
\end{equation}
\begin{formulaDescription}
\item [$W_h$] потужність використовуваного h-ого виду устаткування, кВт;
\item [$T_h$] час роботи h-ого виду устаткування, година;
\item [$K_h$] коефіцієнт використання устаткування;
\item [$T_h$] вартість 1 кВт/год. електроенергії, коп.
\end{formulaDescription}

При розрахунку витрат на електроенергію слід виходити з вартості за 1 кВт/час – 32,00 коп.

Витрати на устаткування і покупні вироби. 

У кошторис включається вартість тільки того устаткування, яке безпосередньо використовується для проведення даного НДР, тобто того, що має одноразове застосування в НДР не передбачається.

Витрати на малоцінний інвентар. 

Витрати на малоцінний інвентар і інструменти, що швидко зношуються, приймають у розмірі 10 – 15 \% вартості використовуваного устаткування. Витрати по цій статті не передбачаються.

Амортизаційні відрахування.

Амортизаційні відрахування розраховуються на основні фонди лабораторії вартістю від 1200 грн., що знаходяться в експлуатації більше одного року.

До таких елементів основних фондів відносять:
\begin{itemize}
\item приміщення;
\item твердий інвентар;
\item устаткування тривалого використання;
\item стенди;
\item вимірювальні прилади.
\end{itemize}

Розрахунок амортизаційних відрахувань (Aм) проводиться по формулі:

\begin{equation}
A_M = \frac{N_a   \cdot   T   \cdot   C_o}{12  \cdot  100}, \nonumber
\end{equation}
\begin{formulaDescription}
\item [$N_a$] норма амортизації основних фондів, \%;
\item [$T$] тривалість виконання НДР, місяць;
\item [$C_o$] вартість основних фондів, грн.
\end{formulaDescription}

Норму амортизації основних фондів слід прийняти в наступних розмірах:

\begin{enumerate}
\item будівлі і споруди --- 5 \%;
\item вимірювальна техніка і інвентар --- 25 \%;
\item устаткування --- 15 \%.
\end{enumerate}

Амортизаційні відрахування обчислювальної техніки, що орендується, включені у вартість 1 години машинного часу. Вартість оренди приміщення оцінюється з розрахунку 654 грн. за 1 м\textsuperscript{2} корисній площі. Вартість устаткування приймається у розмірі 10--15 \% вартості приміщення. 

Виробнича площа лабораторії складає 19 м\textsuperscript{2}, тобто вартість її оренди дорівнює 19 $\cdot$  654 = 12426 грн. Вартість інвентарю складе 12426 $\cdot$  0,15=1863,9 грн.

Амортизаційні відрахування в даному випадку виплачуються в розмірі:

\begin{equation}
A_m = \frac{12426  \cdot  6  \cdot  5+1863,9  \cdot  9  \cdot  25+12480  \cdot  15}{12  \cdot  100} = 816,131. \nonumber
\end{equation}

Накладні витрати включають витрати на загальногосподарські потреби (охорона, опалювання, загальне освітлення і тому подібне). Вони приймаються у розмірі 50\% від фонду заробітної плати, тобто 141600 $\cdot$  0,5=70800 грн. 

Загальна сума витрат по статтях 1-11 складає кошторисну собівартість НДР. Кошторисна собівартість НДР складе:
\begin{equation}
S = 141600+196,53+7992,90+194,50+836,35+816,131 + 70800 = 254382,411 грн	, \nonumber
\end{equation}
Вартість НДР, окрім собівартості, включає планові накопичення, фіксовані податки на прибуток і додану вартість, відрахування до місцевого бюджету. Загальна величина цих добавок складає близько 28\%

Ціна розробки НДР включає собівартість НДР і розрахунковий прибуток в розмірі 8\%, та складе:
\begin{equation}
U=S+S  \cdot  0,08 = 254382,411+254382,411  \cdot  0,08=125874,43 грн, \nonumber
\end{equation}
договірна ціна складе: 
\begin{equation}
\sum = U_{\text{НДР}} + U_{\text{НДР}} \cdot {\text{НДС}} = 125874,43+125874,43  \cdot  0,2=274733 грн. \nonumber
\end{equation}

\subsection{Оцінка соціально-економічного ефекту НДР}
Економічний ефект НДР «Розробка моделей, алгоритмічного та програмного забезпечення для керування бізнес-грою «Маркетинговий канал» відображає ступінь дії результату на сферу матеріального виробництва і споживання. Характер, об'єм і напрям такого впливу різноманітні і можуть бути визначені для різних видів НДР з різною повнотою і ступенем точності.

У загальному вигляді економічний ефект ($E$) пошукових і прикладних наукових досліджень визначається по формулі приведених витрат
\begin{equation}
E = (C_b - C_h) - K_c   \cdot   E_h, \nonumber
\end{equation}
\begin{formulaDescription}
\item [$C_b$] поточні витрати на виробництво до впровадження результатів НДР;
\item [$C_h$] поточні витрати на виробництво після впровадження результатів НДР;
\item [$K_c$] супутні капітальні одноразові витрати, пов'язані з впровадженням НДР;
\item [$E_h$] нормативний коефіцієнт ефективності капітальних вкладень, приймається за 0,15.
\end{formulaDescription}

Для оцінювання соціально-економічного ефекту використовують два види оцінок: перший – це кількісна оцінка соціально-економічного ефекту, другий – якісна оцінка. Зважаючи на те, що дана робота є науково-дослідницькою, то для неї не може бути розраховане чисельне значення оцінки соціально-економічного ефекту. Отже, використаємо якісну оцінку для визначення соціального ефекту науково-дослідної роботи.

Впровадження результатів даної науково-дослідної роботи дозволило б досягти наступних результатів:
\begin{enumerate}
\item розроблений програмний продукт автоматизує процес збору перспективної інформації щодо функціонування маркетингового каналу;
\item підприємці отримають зручний та легкий у використанні інструмент який дозволить планувати роботу каналу та оптимізувати його структуру;
\item використання даної інформаційної системи дозволить збільшити конкурентно-здатність маркетингового каналу за рахунок наявності прогнозів діяльності каналу, що робить планування роботи більш точним.
\end{enumerate}
