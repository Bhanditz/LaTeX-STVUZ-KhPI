\section*{Висновки}\addcontentsline{toc}{section}{\tocsecindent{Висновки}}

У ході даної роботи був проведений аналіз проблеми маркетингової розвідки, а саме, отримання переспективної інформації щодо функціонування маркетингових каналів. Був зроблений вибір, щодо використання імітаційного моделювання маркетингових каналів (ділових, чи бізнес-ігор) як засобу отриманния перспективної інформації. 

Були розроблені: алгоритм побудови імітаційної моделі каналу на основі мереж Петрі, алгоритм проведення імітаційного експерименту (алгоритм ігрового циклу), UML-діаграми для моделювання функціональної структури системи, статичних та динамічних аспектів розроблюваного програмного забезпечення.

Для розробки програмного забезпечення були використані мови Python, JavaScript та Java. В процесі роботи була використана середа розробки IntelliJ IDEA, середа роботи з мережами Петрі PIPE та CASE-засіб для UML-моделювання Visual Paradigm for UML.
