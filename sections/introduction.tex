\section*{Вступ}\addcontentsline{toc}{section}{\tocsecindent{Вступ}}

В сучасному світі, системи виробництва та розповсюдження товарів досягали дуже великих розмірів. Маркетингові канали охоплюють корпорації з сотнями тисяч працівників, і навіть уряди держав. З ростом маркетингових каналів, зростає також грошовий еквівалент вартості помилки в прийнятті рішень щодо функціонування каналів чи їх структури. Тому особливо важливо досліджувати нові методі аналізу таких систем, щоб більш швидко та ефективно знаходити шляхи для оптимізації маркетингового процесу. 

Моделювання систем, зокрема маркетингових каналів дозволяяє отримувати корисну інформацію про їх функціонування без необхідності витрачати ресурси на створення та обслуговування, що є особливо витратним у випадку створення каналів розповсюдження товарів. Відомим методом моделювання економічних систем є ділові ігри, що часто використовуються для освітніх цілей.
